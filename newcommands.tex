\newcommand{\li}{\left|}
\newcommand{\re}{\right|}
\newcommand{\const}{\text{const.}}
\newcommand{\z}{\text}
\newcommand{\terminal}[1]{\colorbox{black}{\textcolor{white}{{\fontfamily{phv}\selectfont \scriptsize{#1}}}}}
\newcommand{\plugin}[1]{\textit{\flq#1\frq}}
\newcommand{\ra}{$\rightarrow$ }
\newcommand{\ar}{\autoref}
\newcommand{\good}[1]{{\usebeamercolor[fg]{title}{\textbf{#1}}}}
\newcommand{\bad}[1]{\textcolor{RedOrange}{\textbf{#1}}}
\newcommand{\dead}[1]{\textcolor{NavyBlue}{\textbf{#1}}}
\newcommand{\cmark}{\good{\ding{51}}}
\newcommand{\xmark}{\bad{\ding{55}}}
\newcommand{\itemfill}{\setlength{\itemsep}{\fill}}
\newcommand{\orderof}[1]{$\mathcal{O}\left(#1\right)$}
\newcommand{\code}[1]{{\footnotesize\dejavu #1}}
\newcommand{\var}[1]{\textcolor{JungleGreen}{#1}}
\newcommand{\meth}[1]{\textcolor{RedOrange}{#1}}
\newcommand{\incl}[1]{\textcolor{YellowGreen!70!RawSienna!100}{#1}}
\newcommand{\str}[1]{\textcolor{MidnightBlue!60!black!100}{#1}}
\newcommand{\tline}[1]{\noalign{\hrule height #1pt}}
\newcommand{\fatwhite}[1]{\multicolumn{1}{c}{\textcolor{white}{\textbf{#1}}}}
\renewcommand{\deg}{$^{\circ}$}
\newcommand{\fpath}[1]{\textbf{\path{#1}}}
\renewcommand\tabularxcolumn[1]{m{#1}}
\newcommand{\alternatecolors}{\rowcolors{2}{YellowOrange!10}{ProcessBlue!10}}
\newcommand{\missfig}[1][.5]{\begin{center} \missingfigure[figheight=#1\textheight,figwidth=#1\textheight, figcolor=white]{}\end{center}}
\newcommand{\caplab}[2]{\ifthenelse{\equal{#1}{nocap}}{}{\caption{#1}}\ifthenelse{\equal{#2}{nolab}}{}{\label{#2}}}
\newcommand{\pic}[3]{\ifthenelse{\equal{#1}{r}}{\includegraphics[width={#2\textheight}, angle=-90]{#3}}{\includegraphics[height={#2\textheight}]{#3}}}
%SUBFIG
\DeclareDocumentCommand{\subfig}{O{.45} O{0} m O{l} m O{nocap} O{nolab}}{\begin{subfigure}{#1\linewidth}\vspace*{#2\textheight}\centering\fig[#4]{#3}{#5}[#6][#7]\end{subfigure}}
%SUBFIGS
\DeclareDocumentCommand{\subfigs}{O{fig} m m O{nocap} O{nolab}}{\begin{figure}[ht!]\centering\ifthenelse{\equal{#1}{fig}}{}{#1\hspace*{.05\linewidth}}#2\hspace*{.05\linewidth}#3\caplab{#4}{#5}\end{figure}}
%FIG
\DeclareDocumentCommand{\fig}{O{n} m m O{nocap} O{nolab}}{\begin{figure}[h!]\centering\pic{#1}{#2}{#3}\caplab{#4}{#5}\end{figure}}
% WRAPFIG
\DeclareDocumentCommand{\wrapfig}{O{l} m m O{nocap} O{nolab}}{
	\begin{wrapfigure}{#1}{#2\linewidth}\includegraphics[width={.95\linewidth}]{#3}\caplab{#4}{#5}\end{wrapfigure}}
\DeclareSIUnit\mhzcm{\mega\hertz\per\centi\meter^2}
\DeclareSIUnit\khzcm{\kilo\hertz\per\centi\meter^2}
\DeclareSIUnit\ncm{n\per\centi\meter^2}
